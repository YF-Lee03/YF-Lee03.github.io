\documentclass{article}
\usepackage{graphicx} % Required for inserting images
\usepackage{hyperref}
\usepackage{amssymb}
\usepackage{amsmath}
\usepackage{amsthm}
\usepackage{geometry}
\usepackage{enumerate}
\usepackage{quiver}
\newtheorem{lemma}{Lemma}
\newtheorem{method}{Method}
\newtheorem{definition}{\bf Definition}
\newtheorem{theorem}{\bf Theorem}
\newtheorem{proposition}{\bf Proposition}
\newtheorem{example}{Example}
\newtheorem{sol}{Solution}
\newtheorem{corollary}{Corollary}


\title{\vspace{-2cm} Dragon Problem}
\author{Yifan Li}
\date{\today}

\begin{document}
\maketitle
The dragon of luck time-travels bringing happiness and prosperity to the year he visits. To jump between the years, he only uses two bi-directional transits $N \leftrightarrow 3N +1$ or $N \leftrightarrow 2N$. Prove that there is a sequence of transits that the dragon can use to come to visit us in 2024, no matter which year he is currently in. (Here we only consider the positive integer)

\begin{definition}
    \begin{equation*}
        \begin{split}
            &T_1:\ \mathbb{Z}^+\rightarrow\mathbb{Z}^+,\ T_1(n)=2n,\\
            &T_2:\ \mathbb{Z}^+\rightarrow\mathbb{Z}^+,\ T_2(n)=3n+1,\\
            &\mathcal{T}=\{T_1,\ T_2\},\\
            &T_3:\ \mathbb{Z}^+\rightarrow\mathbb{Z}^+,\ T_3(n)=2n+1,\\
        \end{split}
    \end{equation*}
\end{definition}

To prove that the dragon can jump to 2024 from any year, we prove the following proposition.

\begin{proposition}
    For $\forall m,\,n \in \mathbb{Z}^+$, there exists a combination $T$ of $T_1$, $T_1^{-1}$, $T_2$ and $T_2^{-1}$ such that $T(m)=n$.
\end{proposition}
\begin{proof}
    To show that an arbitrary positive integer can be transfered to any positive integer, we can show that any positive integer can be transfered to $1$ since the operation we use is in the free group $F_{\mathcal{T}}$ and the inverse of operation is still in $F_{\mathcal{T}}$.\\
    For an even number, we may apply $T_1^{-1}$ to transit it into an odd number. Then we only need to consider the situation of odd number. For an odd number $a$, it has the form $a=2b+1$ for some $b\in\mathbb{N}$. In this case, we have $a=T_3(b)$, if $b$ is still an odd number, we can apply $T_3^{-1}$ again and finally get an even natural number.\\
    We separate the situation into two cases: $0$ and positive even integer $2k$. For the situation of $0$, we may write $a$ as $T_3\circ\ldots\circ T_3(0)=2(2(\ldots(2(0)+1)\ldots)+1)+1=1+2+4+\ldots+2^n$ for some $n\in\mathbb{N}$. For the situation of positive even integer $2k$, we may write $a$ as $T_3\circ\ldots\circ T_3(2k)=2(2(\ldots(2(2k)+1)\ldots)+1)+1=2^{n+2}k+1+2+4+\ldots+2^n$ for some $n\in\mathbb{Z}^+$.\\
    \begin{itemize}
        \item $T_3\circ\ldots\circ T_3(0)$. For $a=1+2+4+\ldots+2^n=2^{n+1}-1$. Applying $T_1^{-1}\circ T_2$ for $n+1$ times, we can get $\frac{3^{n+1}-1}{2}=1+3+9+\ldots+3^{n}$. Notice that $T_2^{-1}(\frac{3^{n+1}-1}{2})=T_2^{-1}(1+3+\ldots+3^{n})=1+3+\ldots+3^{n-1}$. Thus applying $T_2^{-1}$ for $n$ times, we would get $1$.
        \item $T_3\circ\ldots\circ T_3(2k)$. For $a=2^{n+2}k+1+2+4+\ldots+2^n=2^{n+2}k+2^{n+1}-1$. Applying $T_1^{-1}\circ T_2$ for $n+1$ times, we can get $3^{n+1}(2k+1)-1$. Applying $T_1^{-1}$, we have $3^{n+1}k+\frac{3^{n+1}-1}{2}=3^{n+1}k+1+3+\ldots+3^n$. Notice that $T_2^{-1}(3^{n+1}k+1+3+\ldots+3^n)=3^nk+1+3+\ldots+3^{n-1}$. Thus applying $T_2^{-1}$ for $n$ times, we would get $k+1$, which is smaller than $a$. Repeat the previous steps and we can finally reduce to $1$.
    \end{itemize}
\end{proof}




\end{document}