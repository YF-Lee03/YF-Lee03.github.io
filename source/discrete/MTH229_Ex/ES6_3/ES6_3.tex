\documentclass{article}
\usepackage{graphicx} % Required for inserting images
\usepackage{hyperref}
\usepackage{amssymb}
\usepackage{amsmath}
\usepackage{amsthm}
\usepackage{geometry}

\newtheorem{lemma}{Lemma}
\newtheorem{method}{Method}
\newtheorem{definition}{Definition}
\newtheorem{theorem}{\bf Theorem}
\newtheorem{proposition}{\bf Proposition}
\newtheorem{example}{Example}
\newtheorem{sol}{Solution}
\newtheorem{corollary}{Corollary}


\title{\vspace{-2cm} A Solution to Question 3 of Exercise 6}
\author{Yifan Li}
\date{\today}

\begin{document}

\maketitle

\begin{proposition}
    Let $\mu$ be a spectral radius of $G$, then
    \begin{equation*}
        \delta(G)\leq\mu\leq\Delta(G).
    \end{equation*}
\end{proposition}

\begin{proof}
    First we show that $\mu\leq\Delta(G)$. Let $A$ be the adjacent matrix of $G$, since $\mu$ is an eigenvector, thus
    \begin{equation}\label{eq1}
        \exists\, v\,(Av=\mu v).
    \end{equation}
     \eqref{eq1}. 
    \begin{equation}\label{eq2}
        e^TAv=e^T\mu v=\mu e^Tv.
    \end{equation}
    Write $v$ as $[v_1,\,v_2,\,\ldots,\,v_n]$, then the right-hand side can be written as $\mu\sum_{i=1}^nv_i$. For the left-hand side, we have
    \begin{equation}\label{eq3}
        e^TAv=(e^TA^T)v=(Ae)^Tv=\begin{bmatrix}
            \deg(v_1) & \ldots & \deg(v_n)
        \end{bmatrix}v\leq\Delta(G)\,\sum_{i=1}^nv_i.
    \end{equation}
    Thus by Equation \eqref{eq2} and Equation \eqref{eq3}, we have $\mu\leq\Delta(G)$.\\

    Now we try to show that $\mu\geq\delta(G)$. Let $e=\begin{bmatrix}
        1 & 1 & \ldots & 1
    \end{bmatrix}^T$, note that $Ae$ can be regarded as a linear combination of columns $C_i$ of $A$ and the sum of elements of the vector of $C_i$ is $\deg(v_i)$. In this case, we try to study the sum of vector elements in Equation. Note that in the above proof, we have found that $Ae=\begin{bmatrix}
        \deg(v_1) & \ldots & \deg(v_n)
    \end{bmatrix}^T$ and thus the sum of its elements is 
    \begin{equation}\label{eq4}
        e^TAe=\sum_{i=1}^{n}\deg(v_i)=2\,\vert E(G)\vert.
    \end{equation}
    Note that $e$ can also be written as a linear combination of eigenbasis. Let $\{\alpha_1,\,\alpha_2,\,\ldots,\,\alpha_n\}$ be an orthonormal eigenbasis of $A$, then
    \begin{equation*}
        \exists r_1,\,r_2,\,\ldots,\,r_n\,(e=\sum_{i=1}^{n}r_i\alpha_i).
    \end{equation*}
    Then
    \begin{equation}\label{eq5}
        \begin{split}
            e^TAe&=(\sum_{i=1}^{n}r_i\alpha^T_i)\left(A(\sum_{i=1}^{n}r_i\alpha_i)\right)=(\sum_{i=1}^{n}r_i\alpha^T_i)(\sum_{i=1}^{n}\lambda_i r_i\alpha_i)=\sum_{i=1}^{n}\lambda_ir_i^2\\
            &\leq\mu\sum_{i=1}^{n}r_i^2=\mu e^Te=\mu n.
        \end{split}
    \end{equation}
    By Equation \eqref{eq4} and Equation \eqref{eq5}, one may find that 
    \begin{equation*}
        \mu n \geq 2 \vert E(G) \vert \geq n \delta(G).
    \end{equation*}
    Thus $\mu\geq\delta(G)$.
\end{proof}

\end{document}